\documentclass{article}
\usepackage[sorting=none]{biblatex}
\addbibresource{report.bib}
\title{CS890BR Project}
\date{\today}
\author{Riley Herman 200352833}

\begin{document}
    \pagenumbering{gobble}
    \maketitle
    \begin{abstract}
        this is the abstract. you know
    \end{abstract}
    \newpage
    \pagenumbering{arabic}
    \section{Introduction}
    In the early days of the stock market, William Stanley Jevons developed a widely used theory on how to predict changes in the market.
    This is the man that wrote such influential books as The Theory of Political Economy, expounding the marginal utility theory of value that 
    is fundamental to understanding modern economics. A first car has enormous value in comparison to a fifteenth car. Unfortunately for Jevons, 
    his market prediction theory was not so revolutionary. William Peter Hamilton recounts, "[Jevons] propounded the theory of a connection between 
    commercial panics and spots on the sun. \ldots I said that while Wall Street in its heart believed in a cycle fo panic and prosperity, 
    it did not care if there were enough spots of the sun to make a straight flush" \cite{Hamilton}. As much as Jevons' sun spot prediction method
    sounds absurd, the inherent nature of the stock market is to be unpredictable. His theory is every bit as valid as any other when the theory 
    attempts to explain a complex adaptive system. Michael Mauboussin acknowledges that "[m]ost systems, in nature and in business,
    are not in equilibrium but rather in constant flux" \cite{Mauboussin}. \\
    Having said that, and despite the orange tabby cat Orlando's success as a stock picker \cite{King}, there is little disagreement that making
    informed portfolio choices is in the best interest of the investor. Therefore, using historical data to generate a set of optimal portfolios
    is the widely accepted approach. Given the multitude of solving techniques, it is in the investor's best interest to determine the most efficient solver
    to find the most efficient solution. There are five algorithms presented in this paper: Flower Pollination, Non-dominated Sorting Genetic 
    Algorithm 2 (hereby referred to as NSGA2), Particle Swarm Optimization (PSO), Strength Pareto Evolutionary Algorithm 2 (SPEA2), and traditional
    Branch and Bound, an exact algorithm the other algorithms are to be compared against \cite{Yang} \cite{KaucicMoradiMirzazadeh}. 
    \section{Background Knowledge}
    In order to understand the recommended portfolios and how they are arrived at, one must first understand the problem that is being solved, how it is 
    modelled, what techniques are available and why they were chosen. 
    \subsection{Portfolio Optimization}
    The problem of portfolio optimization often finds its roots in Harry Markowitz's frequently cited book Portfolio Selection \cite{Markowitz}. Though
    the language may not be the same, what he is proposing is clearly a multi-objective constraint optimization problem. He suggests that portfolios
    should be weighted using risk and return with respect to a budgetary constraint. Though the problem originates in the field of 
    economics, computing has approached the problem as an excellent candidate for a variety of multi-objective constraint optimization algorithms. \\
    The problem is formulated as follows: given a set of stocks and a budget, one must use the risk and reward to find an amount of each stock to
    purchase such that risk is minimal and reward is maximal. Risk and reward are not comparable to each other; the algorithm cannot decide whether
    a conservative, low reward but also low risk portfolio is better than an aggressive, high risk but high reward portfolio. Thus the solution is a 
    set of options, each one best in its own way. There are several pieces of information that need to be calculated: prices, which are often taken 
    directly from source data; reward, which is NEED TO FILL IN HOW REWARD WORKS; and risk, which was once upon a time calculated as variance
    but is now often calculated as value-at-risk. Value at risk is a type of downside risk; that is, it attempts to measure risk by estimating the
    maximum amount lost on the stock \cite{HongHuZhang} \cite{Cid}. There are a variety of other objectives that could be optimized towards as well: 
    liquiditiy, tax efficiency, etc. However, since the premise of these quantitative optimizations are the same it has been left out of the scope 
    of this project. Another set of objectives that could be optimized towards are qualitative; examples of this may be longevity of the portfolio,
    where a longer term portfolio may be able to accept more risk than a shorter term portfolio \cite{Xiongetal}, or social responsibility of the 
    companies being traded. Once the problem is formulated in terms of variables (the stocks and how much to buy of each one), constraints (the 
    total amount invested must be below the budget), and objective functions (risk and reward), the problem is ready to be modelled as a constraint
    optimization problem.
    \subsection{Constraint Optimization Techniques}
    cop
    \section{Implementation}
    This section describes details of the implementation of the python library, as well as the data captured
    \section{Conclusion}
    This is the conclusion where we recap all the fun we had this past week.
    \newpage
    \printbibliography
\end{document}