\documentclass{article}
\usepackage[sorting=none]{biblatex}
\addbibresource{proposal.bib}
\title{Project Proposal}
\date{2020-07-03}
\author{Riley Herman 200352833}

\begin{document}
\pagenumbering{gobble}
\maketitle
\newpage
\pagenumbering{arabic}
Portfolio optimization is a widely studied multi-objective optimization problem. Many cite Markowitz's
paper on portfolio selection as one of the early research papers on the topic. In it he describes
a similar type of optimization that I will be proposing for this project: one built on maximizing
returns and minimizing risks. \cite{Markowitz} \\
The project I would like to do for CS890BR is a portfolio optimization algorithm comparison.
I will implement several algorithms targetted at portfolio optimization and compare them based
on time and quality of the solution set. The solution set of portfolios will be multi-objective;
for this iteration the only two objectives that will be considered are risk, or value-at-risk,
and reward, or a return as a rate over a similar period of time.
I will be using an arbitrary set of Canadian stocks and using their historical data over the past 90 days
as an input to the program. The program will output a set of pareto optimal solutions found by several
algorithms. \\
The first step of the implementation will be to calculate the projected variance and return
from the historical data. Using this new set of data, the program will attempt to find appropriate
solutions. The data conversion simply takes the mean of the daily historical data to compute rate
of return and risk in terms of value-at-risk, or VaR. VaR is a type of downside risk; that is,
it attempts to measure risk by estimating the maximum amount lost on the stock \cite{HongHuZhang} \cite{Cid}.
There are a variety of other objectives that could be optimized towards as well: liquiditiy, tax
efficiency, etc. However, since the premise of these quantitative optimizations are the same it is 
out of the scope of this project. Another set of objectives that could be optimized towards are qualitative; 
examples of this may be longevity of the portfolio, where a longer term portfolio may be able to accept
more risk than a shorter term portfolio \cite{Xiongetal}, or social responsibility. These are also left 
out of the scope of this project intentionally because of the complexity involved. \\
The next step of the implementation is to solve the multi-objective constraint optimization problem.
This will be done using six distinct algorithms: five genetic algorithms and an exact method, branch
and bound, as a control to compare solutions and execution time. The genetic algorithms that will be
used are NSGA-2 (Non-dominated Sorting Genetic Algorithm 2), SPEA-2 (Strength Pareto Evolutionary
Algorithm 2), PSO (Particle Swarm Optimization), Flower Pollination, and Bee Colony Optimization.
These have been selected for a variety of reasons; some are very common and relied upon, such as
NSGA-2 and PSO \cite{KaucicMoradiMirzazadeh}, and others are newer and not as well documented in multi-objective problem solving,
such as Flower Pollination. \cite{Yang} \\
Algorithms will be compared using a statistical analysis of the runtime over several attempts at the
problem. Portfolio quality will be compared using their differences from the exact solution found by
the branch and bound algorithm. \\

\printbibliography
\end{document}